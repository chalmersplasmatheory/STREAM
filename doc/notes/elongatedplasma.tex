\documentclass{notes}

\title{Elongated plasma geometry for STREAM}
\author{Mathias Hoppe}

\newcommand{\Rm}{R_{\rm m}}
\newcommand{\Rhat}{\hat{\bb{R}}}
\newcommand{\Jac}{\mathcal{J}}

\begin{document}
    \maketitle

    \noindent
    The \DYON\ code uses an elongated flux plasma geometry, with a time evolving
    plasma minor radius $a$ and elongation $\kappa$. In this document, we derive
    expressions for the corresponding spatial Jacobian required by \STREAM. The
    discussion is based on that of \texttt{doc/notes/theory.pdf} in the \DREAM\
    repository.

    \section{Definitions}
    Similar to the analytical magnetic field of \DREAM, we parametrize flux
    surfaces according to
    \begin{equation}\label{eq:fsparam}
        \begin{aligned}
            \bb{x} &= R\Rhat + z\zhat,\\
            %
            R &= \Rm + r\cos\theta,\\
            %
            z &= r\kappa(r)\sin\theta,\\
            %
            \Rhat &= \cos\varphi\xhat + \sin\varphi\yhat,
        \end{aligned}
    \end{equation}
    where $\bb{x}$ is the regular position vector, $R$ is the major radius
    coordinate, $\Rm$ is the tokamak/plasma major radius, $z$ is the cartesian
    vertical coordinate, $\varphi$ is the toroidal angle, $r$ is the minor
    radius coordinate, $\theta$ is the poloidal angle, and $\kappa(r)$ is the
    elongation.

    Formally, quantities in \STREAM\ are flux-surface averaged, meaning that
    they are integrated over the poloidal and toroidal angles. We denote a
    flux-surface average using angle-bracket notation:
    \begin{equation}
        \begin{aligned}
            \left\langle X \right\rangle &=
                \frac{1}{V'}\int_0^{2\pi}\dd\theta\int_0^{2\pi}\dd\varphi\,\Jac\,X,\\
            %
            V' &= \frac{1}{V'}\int_0^{2\pi}\dd\theta\int_0^{2\pi}\dd\varphi\,\Jac.\\
        \end{aligned}
    \end{equation}
    Here, $\Jac$ denotes the Jacobian for the toroidal coordinate system
    $(r,\theta,\varphi)$.

    \section{Geometrical quantities}
    We will consider an elongated plasma with constant elongation
    $\kappa(r)=\kappa_0$. In such a plasma, the Jacobian $\Jac$ becomes
    \begin{equation}
        \Jac = \kappa r R\left(
            1 + \frac{r\kappa'}{\kappa}\sin^2\theta
        \right) = \kappa rR,
    \end{equation}
    where $\kappa'(r) = \partial\kappa/\partial r = 0$. Aside from the Jacobian,
    we also need to know 
    \begin{equation}
        \left|\nabla r\right|^2 = \frac{\kappa^2r^2R^2}{\Jac^2}\left[
            \cos^2\theta + \frac{1}{\kappa^2}\sin^2\theta
        \right] =
        \cos^2\theta + \frac{1}{\kappa^2}\sin^2\theta.
    \end{equation}
    We note that the quantity $R/\Rm$, which is also needed by \STREAM,
    evaluates to $R/\Rm=1$ in the large-aspect ratio limit ($a\ll\Rm$) which we
    work in.

    \section{SPI quantities}
    For SPI support, we also need to convert between a cartesian and toroidal
    basis for some quantities. The first conversion needed is simply the
    inverse of equation~\eqref{eq:fsparam}, taking us from cartesian $(x,y,z)$
    to toroidal coordinates $(r,\theta)$ (note that the SPI module substitutes
    $z\to y$ and lets $z$ denote the toroidal direction). The relation between
    $(r,\theta)$ and $(x,y)$ is
    \begin{equation}
        \begin{aligned}
            r &= \sqrt{\frac{x^2+y^2}{\cos^2\theta+\kappa^2\sin^2\theta}},\\
            \tan\theta &= \frac{y}{2\kappa x}.
        \end{aligned}
    \end{equation}
    With
    \begin{equation}
        \sin\theta = \frac{y}{\sqrt{\kappa^2x^2 + y^2}},\quad
        \cos\theta = \frac{\kappa x}{\sqrt{\kappa^2x^2 + y^2}},
    \end{equation}
    we can simplify $r$ to
    \begin{equation}\label{eq:cartr}
        r = \frac{\sqrt{\kappa^2x^2 + y^2}}{\kappa}.
    \end{equation}
    This in turn allows us to express the trigonometric functions in terms of
    $r$ as
    \begin{equation}
        \sin\theta = \frac{x}{r},\quad \cos\theta = \frac{y}{\kappa r}.
    \end{equation}

    We will also need to express the gradient of $r$ in cartesian coordinates.
    From equation~\eqref{eq:cartr} we find that the non-vanishing components of
    $\nabla r$ become
    \begin{equation}
        \begin{aligned}
            \frac{\partial r}{\partial x} &= \frac{\kappa x}{r} = \cos\theta,\\
            %
            \frac{\partial r}{\partial y} &= \frac{y}{\kappa r} = \kappa\sin\theta.
        \end{aligned}
    \end{equation}
    Finally, \ldots

\end{document}

