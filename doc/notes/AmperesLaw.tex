\documentclass{notes}

\author{Mathias Hoppe}
\title{A direct solution of Ampère's law}

\begin{document}
    \maketitle

    \noindent
    In \DREAM, the initial value of the poloidal flux $\psi$ is calculated
    from an inverted Amp\'ere's law. In other words, the Amp\'ere's law which
    is usually solved in \DREAM\ is rewritten on an integral form and evaluated
    numerically. While this is analytically consistent, the numerical solutions
    to the differential and integral forms of the equation are generally
    different. This in turn gives rise to a mismatch in the electric field and
    poloidal flux when the latter is eventually being evolved according to the
    differential form of Amp\'ere's law in the main part of the simulation.

    The implications of this for disruption simulations are minor since
    $E\approx 0$ at the start of the simulation (at least compared to the
    electric field in the current quench). However, in the startup simulations
    conducted with \STREAM, the exact value of the initial electric field
    could potentially play an important role. In \STREAM\ it might therefore
    be crucial to solve Amp\'ere's law consistently for both the initial value
    of $\psi$ as well as its time evolution.

    In this document we derive the solution to the differential form of
    Amp\'ere's law in the special case $N_r = 1$ (one radial grid point), which
    could be straightforwardly implemented in \DREAM\ and improve the
    solution accuracy.

    \section*{Derivation}
    Amp\'ere's law, as stated in \DREAM\ and \STREAM, is
    \begin{equation}\label{eq:ampere}
        \frac{1}{V'}\frac{\partial}{\partial r}\left[
            V'\left\langle \frac{\left|\nabla r\right|^2}{R^2} \right\rangle
            \frac{\partial\psi}{\partial r}
        \right]
        = 2\pi\mu_0\left\langle\bb{B}\cdot\nabla\phi\right\rangle\frac{j_{\rm tot}}{B},
    \end{equation}
    where $V'$ is the spatial jacobian, $r$ the minor radius coordinate, $R$ the
    major radius coordinate, $\bb{B}$ the magnetic field vector, $\phi$ the
    toroidal angle coordinate, and $j_{\rm tot}$ the plasma current density.

    We proceed by discretizing equation~\eqref{eq:ampere} in the special case of
    one radial grid point. According to equation~(56) of the \DREAM\ paper,
    equation~\eqref{eq:ampere} discretizes as
    \begin{equation}\label{eq:discr}
        \frac{1}{V_1'\Delta r_1}\left[
            V_{3/2}'
            \left.\left\langle\frac{\left|\nabla r\right|^2}{R^2}\right\rangle\right|_{3/2}
            \left.\frac{\partial\psi}{\partial r}\right|_{3/2}
            -
            V_{1/2}'
            \left.\left\langle\frac{\left|\nabla r\right|^2}{R^2}\right\rangle\right|_{1/2}
            \left.\frac{\partial\psi}{\partial r}\right|_{1/2}
        \right]
        = 2\pi\mu_0\left.\left\langle\bb{B}\cdot\nabla\phi\right\rangle\right|_{1}
        \frac{j_1}{B(r_1)}.
    \end{equation}
    The second term on the LHS vanishes identically, either by noting that
    $V_{1/2}' = 0$, or by noting that because of symmetry
    $\partial\psi/\partial r = 0$ in the center of the plasma, $r=r_{1/2}$.
    The remaining $r$-derivative can be evaluated with the help of
    equation~(59) in the \DREAM\ paper, but with $r_{2}$ (which is so far
    undefined in this single-radial-grid-point system) taken as the plasma
    edge:
    \begin{equation}\label{eq:deriv}
        \left.\frac{\partial\psi}{\partial r}\right|_{3/2} =
        \frac{\psi_2-\psi_1}{\Delta r_{3/2}}.
    \end{equation}
    Taking $r_2=a$, with $a$ the plasma minor radius, we obtain
    \begin{equation}
        \begin{aligned}\label{eq:edge}
            \psi_2 &\equiv \psi_{\rm edge},\\
            \Delta r_{3/2} &\equiv a-r_1.
        \end{aligned}
    \end{equation}
    Plugging equations~\eqref{eq:deriv} and~\eqref{eq:edge} into
    equation~\eqref{eq:ampere} we obtain the following algebraic expression for
    $\psi_1$:
    \begin{equation}
        \left.V_{3/2}'\left\langle\frac{\left|\nabla r\right|^2}{R^2}\right\rangle\right|_{3/2}
        \frac{\psi_{\rm edge} - \psi_1}{(a-r_1)V_1'\Delta r_1}
        =
        \left.2\pi\mu_0\left\langle \bb{B}\cdot\nabla\phi \right\rangle\right|_1
        \frac{j_1}{B(r_1)}.
    \end{equation}
    Finally, solving for $\psi_1$ we arrive at
    \begin{equation}
        \psi_1 = \psi_{\rm edge} -
        \frac{
            \left.2\pi\mu_0\left\langle \bb{B}\cdot\nabla\phi \right\rangle\right|_1
            (a-r_1)V_1'\Delta r_1
        }{
            \left.V_{3/2}'\left\langle\frac{\left|\nabla r\right|^2}{R^2}\right\rangle\right|_{3/2}
            B(r_1)
        } j_1.
    \end{equation}
\end{document}
